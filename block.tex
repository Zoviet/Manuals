\documentclass[a4paper,10pt,landscape]{article}
\usepackage{luacode}
\usepackage{polyglossia} % Поддержка многоязычности (fontspec подгружается автоматически)
\setmainlanguage[babelshorthands=true]{russian} % Язык по-умолчанию русский с поддержкой приятных команд пакета babel
\setotherlanguage{english} % Дополнительный язык = английский (в американской вариации по-умолчанию)
\setmonofont{CMU Typewriter Text} % моноширинный шрифт
\newfontfamily\cyrillicfonttt{CMU Typewriter Text} % моноширинный шрифт для кириллицы
\defaultfontfeatures{Ligatures=TeX} % стандартные лигатуры TeX, замены нескольких дефисов на тире и т. п. Настройки моноширинного шрифта должны идти до этой строки, чтобы при врезках кода программ в коде не применялись лигатуры и замены дефисов
\setmainfont{CMU Serif} % Шрифт с засечками
\newfontfamily\cyrillicfont{CMU Serif} % Шрифт с засечками для кириллицы
\setsansfont{CMU Sans Serif} % Шрифт без засечек
\newfontfamily\cyrillicfontsf{CMU Sans Serif} % Шрифт без засечек для кириллицы
\usepackage{circuitikz}
\usepackage{amsmath,graphicx}
\usepackage[utf8x]{inputenc}
\usepackage{multicol,multirow}
\usepackage{lmodern}
\usepackage{amsmath}
\usepackage{amsfonts}
\usepackage{subfigure}
\usepackage{textpos}
\usepackage{verbatim}
\usepackage[colorlinks=true,citecolor=blue,linkcolor=blue]{hyperref}
\usepackage{geometry}
\usepackage{tabularx}
\usepackage{blindtext}

\paperwidth = 297mm
\geometry{top=0.5cm,left=0.5cm,right=0.5cm,bottom=0.5cm} 

\usepackage{enumitem}
\setlist[itemize]{noitemsep,nolistsep}

\pagestyle{empty}
\makeatletter
\renewcommand{\section}{\@startsection{section}{1}{0mm}%
                                {-1ex plus -.5ex minus -.2ex}%
                                {0.5ex plus .2ex}%x
                                {\normalfont\large\bfseries}}
\renewcommand{\subsection}{\@startsection{subsection}{2}{0mm}%
                                {-1explus -.5ex minus -.2ex}%
                                {0.5ex plus .2ex}%
                                {\normalfont\normalsize\bfseries}}
\renewcommand{\subsubsection}{\@startsection{subsubsection}{3}{0mm}%
                                {-1ex plus -.5ex minus -.2ex}%
                                {1ex plus .2ex}%
                                {\normalfont\small\bfseries}}
\makeatother


% -----------------------------------------------------------------------

\begin{document}

\raggedright
\footnotesize

\begin{center}
     \Large{\textbf{Модуль управления СКУД контроллерами hh24block}} \\
\end{center}
\begin{multicols}{3}
\setlength{\premulticols}{1pt}
\setlength{\postmulticols}{1pt}
\setlength{\multicolsep}{1pt}
\setlength{\columnsep}{2pt}

\section{Варианты исполнения}

\noindent\begin{tabular}{p{1cm}|p{3cm}|p{3cm}}
\hline
\texttt{}&\texttt{Подключение модуля hh24sens}&\texttt{Подключение модуля hh24relay}\\
\hline
\textbf{S1}&\multicolumn{1}{c}{\textbf{-}}&\multicolumn{1}{c}{\textbf{-}}\\
\textbf{S2}&\multicolumn{1}{c}{\textbf{+}}&\multicolumn{1}{c}{\textbf{-}}\\
\textbf{S3}&\multicolumn{1}{c}{\textbf{-}}&\multicolumn{1}{c}{\textbf{+}}\\
\textbf{S4}&\multicolumn{1}{c}{\textbf{+}}&\multicolumn{1}{c}{\textbf{+}}\\
\hline
\end{tabular}

\section{Функциональные возможности}

\begin{itemize} 
  \item управление контроллерами СКУД c протоколом Weigand 26|34 через сервер hh24lock* и внешнее API  
  \item обеспечение резервированного питания контроллера СКУД и электромеханического замка
  \item поддержка модуля внешних датчиков hh24sens (варианты S2 и S4)
  \item поддержка модуля внешних силовых реле hh24relay (варианты S3 и S4)
\end{itemize}
* Поддерживаются все контроллеры с наличием выхода Weigand и входом для кнопки открытия двери изнутри

\section{Технические характеристики}

\noindent\begin{tabular}{p{6cm}|p{2cm}|p{2cm}}
\hline
\texttt{Напряжение питания}&\multicolumn{1}{c}{\texttt{110-220}}&\multicolumn{1}{c}{\texttt{В}}\\
\hline
\texttt{Ток потребления}&\multicolumn{1}{c}{\texttt{<0.18}}&\multicolumn{1}{c}{\texttt{мА}}\\
\hline
\texttt{Напряжение на выходе питания контроллера}&\multicolumn{1}{c}{\texttt{11.4}}&\multicolumn{1}{c}{\texttt{В}}\\
\hline
\texttt{Напряжение на выходе питания релейного модуля}&\multicolumn{1}{c}{\texttt{3.4-4.2}}&\multicolumn{1}{c}{\texttt{В}}\\
\hline
\texttt{Максимальный ток на выходе 12 В}&\multicolumn{1}{c}{\texttt{2.1}}&\multicolumn{1}{c}{\texttt{А}}\\
\hline
\texttt{Максимальный ток на выходе 3.4 В}&\multicolumn{1}{c}{\texttt{50}}&\multicolumn{1}{c}{\texttt{мА}}\\
\hline
\texttt{Безтоковая пауза при исчезновении питания от сети, не более}&\multicolumn{1}{c}{\texttt{150}}&\multicolumn{1}{c}{\texttt{мс}}\\
\hline
\texttt{Безтоковая пауза при выходе из автономного режима (появления напряжения сети)}&\multicolumn{1}{c}{\texttt{0}}&\multicolumn{1}{c}{\texttt{с}}\\
\hline
\texttt{Время автономной работы в режиме ожидания, не менее}&\multicolumn{1}{c}{\texttt{48}}&\multicolumn{1}{c}{\texttt{час}}\\
\hline
\texttt{Количество срабатываний замка за 20 сек в автономном режиме до включения защиты, не менее}&\multicolumn{1}{c}{\texttt{4}}&\multicolumn{1}{c}{\texttt{раз}}\\
\hline
\texttt{Время полной зарядки аккумуляторов после цикла автономной работы в течении 24 часов, не более}&\multicolumn{1}{c}{\texttt{3}}&\multicolumn{1}{c}{\texttt{час}}\\
\hline
\texttt{Поддерживаемые протоколы Weigand}&\multicolumn{1}{c}{\texttt{26,34}}&\multicolumn{1}{c}{\texttt{}}\\
\hline
\end{tabular}

\section{Функции защиты}

\begin{itemize} 
  \item защита от перезарядки встроенных аккумуляторов    
  \item защита от повреждения встроенных аккумуляторов
  \item защита от короткого замыкания на выходе
  \item защита от превышения максимально допустимого тока
\end{itemize}

\textbf{При срабатывании любой функции защиты во время работы устройства в автономном режиме самостоятельное восстановление работы устройства возможно только после подключения его к питающей сети}


\section{Подключение к контроллеру СКУД}

Подключение контроллера СКУД к устройству осуществляется по инструкции контроллера в соответствии со схемой подключения производителя для режима Weigand Output Reader или аналогичного. Для большинства распространенных контроллеров Tantos, PS-Link и т.д. цветовая маркировка в жгуте проводов будет следующей:

\noindent\begin{tabular}{p{3cm}|p{2cm}|p{2cm}}
\hline
\texttt{}&\texttt{Цвет провода hh24block}&\texttt{Цвет провода контоллера}\\
\hline
\texttt{+12 В}&\multicolumn{1}{c}{\texttt{Светло-красный}}&\multicolumn{1}{c}{\texttt{Красный}}\\
\hline
\texttt{GND}&\multicolumn{1}{c}{\texttt{Серый}}&\multicolumn{1}{c}{\texttt{Черный}}\\
\hline
\texttt{D0}&\multicolumn{1}{c}{\texttt{Зеленый}}&\multicolumn{1}{c}{\texttt{Зеленый}}\\
\hline
\texttt{D1}&\multicolumn{1}{c}{\texttt{Белый}}&\multicolumn{1}{c}{\texttt{Белый}}\\
\hline
\texttt{OPENDOOR*}&\multicolumn{1}{c}{\texttt{Желтый}}&\multicolumn{1}{c}{\texttt{Желтый}}\\
\hline
\end{tabular}

*Провод OPENDOOR должен быть подключен со стороны контроллера СКУД, а не кнопки открытия двери:

\begin{circuitikz}
\draw
	(-1,0.5) node[ground] {} 	
	to[short,o-] node[left] {} (0,0.5)	
	to[short,o-] (3,0.5)	
	node[label={[font=\footnotesize]right:$GND$}] {}
	to[short,o-] (0,0.5)	
	to[push button] node[label={}] {} (0,1.5)	
	to[short,o-] (-1,1.5)	
	node[label={[font=\footnotesize]left:$OPEN$}] {} 
	to[short,o-] (0,1.5)
	to[short,o-] (3,1.5)
	node[label={[font=\footnotesize]right:$OPENDOOR$}] {} 
	to[short,o-] (0,1.5)
	(-1,2) node[label={[font=\footnotesize]left:$+12$}] {} 
	to[short,o-] (3,2)	
	node[label={[font=\footnotesize]right:$+12$}] {} 
	to[short,o-] (-1,2)	
	(-1,2.5) node[label={[font=\footnotesize]left:$D1$}] {} 
	to[short,o-] (3,2.5)	
	node[label={[font=\footnotesize]right:$D1$}] {} 
	to[short,o-] (-1,2.5)	
	(-1,3) node[label={[font=\footnotesize]left:$D0$}] {} 
	to[short,o-] (3,3)	
	node[label={[font=\footnotesize]right:$D0$}] {} 
	to[short,o-] (-1,3)	
;

\end {circuitikz}

\section{Подключение к модулю датчиков hh24sens}

Доступно для вариантов исполнения S2 и S4.

\noindent\begin{tabular}{p{3cm}|p{2cm}|p{2cm}}
\hline
\texttt{}&\texttt{Цвет провода hh24block}&\texttt{Вход модуля}\\
\hline
\texttt{+3.3 В}&\multicolumn{1}{c}{\texttt{Красный}}&\multicolumn{1}{c}{\texttt{+3.3}}\\
\hline
\texttt{GND}&\multicolumn{1}{c}{\texttt{Коричневый}}&\multicolumn{1}{c}{\texttt{GND}}\\
\hline
\texttt{A0}&\multicolumn{1}{c}{\texttt{Желтый}}&\multicolumn{1}{c}{\texttt{Sound}}\\
\hline
\texttt{D0}&\multicolumn{1}{c}{\texttt{Синий}}&\multicolumn{1}{c}{\texttt{Motion}}\\
\hline
\end{tabular}

\section{Подключение к модулю реле hh24relay}

Доступно для вариантов исполнения S3 и S4.

\noindent\begin{tabular}{p{3cm}|p{2cm}|p{2cm}}
\hline
\texttt{}&\texttt{Цвет провода hh24block}&\texttt{Вход модуля}\\
\hline
\texttt{+3.3 В}&\multicolumn{1}{c}{\texttt{Красный}}&\multicolumn{1}{c}{\texttt{+3.3}}\\
\hline
\texttt{GND}&\multicolumn{1}{c}{\texttt{Коричневый}}&\multicolumn{1}{c}{\texttt{GND}}\\
\hline
\texttt{R1}&\multicolumn{1}{c}{\texttt{Желтый}}&\multicolumn{1}{c}{\texttt{Relay1}}\\
\hline
\texttt{R1}&\multicolumn{1}{c}{\texttt{Синий}}&\multicolumn{1}{c}{\texttt{Relay2}}\\
\hline
\end{tabular}

\section{Первое включение устройства}

При первом включении устройство сохраняет для дальнешей работы сведения о Wi-Fi сети, с которой оно будет работать. Для успешного подключения понадобится пароль от Wi-Fi сети и любое другое устройство, имеющее возможность работы с Wi-Fi (например, мобильный телефон или ноутбук).

Для осуществления привязки необходимо:

\begin{itemize}
  \item подключить устройство к контроллеру СКУД согласно схеме выше 
  \item удостовериться в наличии и работоспособности Wi-Fi сети, с которой будет работать устройство
  \item осуществить принудительную перезагрузку устройства 
  \item на втором устройстве в списке сетей Wi-Fi выбрать сеть hh24lock и подключиться к ней
  \item подключение потребует дополнительной авторизации - на мобильном телефоне или ноутбуке появится соответствующее уведомление. В случае отсутствия уведомления можно открыть в браузере сайт https://hh24lock.ru/ 
  \item в появившемся диалоговом окне необходимо выбрать Wi-Fi сеть, с которой будет работать устройство, и ввести пароль к ней
  \item надпись Success укажет на успешное подключение устройства. При этом из списка доступных Wi-Fi сетей исчезнет сеть hh24lock.
\end{itemize}

В случае смены рабочей Wi-Fi сети и/или пароля к ней, устройство потребуется переподключить. 

\section{Принудительная перезагрузка устройства}

Принудительная перезагрузка устройства возможна двумя способами:  

\begin{itemize} 
  \item путем кратковременного (не более, чем на 10 секунд) отключения устройства от питающей сети 
  \item путем нажатия тонким предметом на кнопку сброса, расположенную за наклейкой со стороны выводного провода
\end{itemize}

Большинство подключенных СКУД контроллеров подтвердят успешность перезагрузки устройства зеленым сигналом светодиода. 

\section{Возможные неполадки и способы их устранения}

\noindent\begin{tabular}{p{4cm}|p{4cm}}
\hline
\texttt{Нет напряжения на выходе в автономном режиме}&\texttt{Кратковременно подключить устройство к сети}\\
\hline
\texttt{Устройство не заряжается в процессе работы, но заряжается в режиме ожидания}&\texttt{Отключить устройство от сети на 10-15 минут и включить заново}\\
\hline
\texttt{Устройство не работает в автономном режиме}&\texttt{Заменить аккумуляторы}\\
\hline
\texttt{Устройство не передает данные и/или не получает коды открытия двери с сервера}&\texttt{Переподключить устройство к Wi-Fi сети как описано в разделе "Первое включение устройства"}\\
\hline
\end{tabular}

\section{Замена аккумуляторов}

\textbf{Замена аккумуляторов возможна только парами.} Необходимо использовать аккумуляторы типа 18650, поддерживающие высокотоковый разряд. 

Замену необходимо производить на выключенном от сети устройстве. После замены аккумуляторов сработает защита и устройство автоматически отключится. Для проверки его работоспособности с замененными аккумуляторами необходимо кратковременно включить устройство в сеть. 

\begin{luacode}
local user = require("user")
	function get_login(id)
		tex.print(user.get_login(id))
	end
	function get_password(id)
		tex.print(user.get_password(id))
	end
\end{luacode}

\section{Доступ к hh24lock.ru и API}

\fbox{ 
	ID устройства: \textbf{\jobname }
	Логин hh24lock.ru: \textbf{\luadirect{get_password(\luastringN{\jobname})} }
	Пароль: \textbf{\luadirect{get_login(\luastringN{\jobname})} }
}

\end{multicols}
\end{document}



