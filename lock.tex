\documentclass[a4paper,10pt,landscape]{article}
\usepackage{luacode}
\usepackage{polyglossia} % Поддержка многоязычности (fontspec подгружается автоматически)
\setmainlanguage[babelshorthands=true]{russian} % Язык по-умолчанию русский с поддержкой приятных команд пакета babel
\setotherlanguage{english} % Дополнительный язык = английский (в американской вариации по-умолчанию)
\setmonofont{CMU Typewriter Text} % моноширинный шрифт
\newfontfamily\cyrillicfonttt{CMU Typewriter Text} % моноширинный шрифт для кириллицы
\defaultfontfeatures{Ligatures=TeX} % стандартные лигатуры TeX, замены нескольких дефисов на тире и т. п. Настройки моноширинного шрифта должны идти до этой строки, чтобы при врезках кода программ в коде не применялись лигатуры и замены дефисов
\setmainfont{CMU Serif} % Шрифт с засечками
\newfontfamily\cyrillicfont{CMU Serif} % Шрифт с засечками для кириллицы
\setsansfont{CMU Sans Serif} % Шрифт без засечек
\newfontfamily\cyrillicfontsf{CMU Sans Serif} % Шрифт без засечек для кириллицы
\usepackage{circuitikz}
\usepackage{amsmath,graphicx}
\usepackage[utf8x]{inputenc}
\usepackage{multicol,multirow}
\usepackage{lmodern}
\usepackage{amsmath}
\usepackage{amsfonts}
\usepackage{subfigure}
\usepackage{textpos}
\usepackage{verbatim}
\usepackage[colorlinks=true,citecolor=blue,linkcolor=blue]{hyperref}
\usepackage{geometry}
\usepackage{tabularx}
\usepackage{blindtext}
\usepackage{luapackageloader}

\paperwidth = 297mm
\geometry{top=1cm,left=1cm,right=1cm,bottom=1cm} 
\geometry{top=1cm,left=1cm,right=1cm,bottom=1cm} 

\usepackage{enumitem}
\setlist[itemize]{noitemsep,nolistsep}

\pagestyle{empty}
\makeatletter
\renewcommand{\section}{\@startsection{section}{1}{0mm}%
                                {-1ex plus -.5ex minus -.2ex}%
                                {0.5ex plus .2ex}%x
                                {\normalfont\large\bfseries}}
\renewcommand{\subsection}{\@startsection{subsection}{2}{0mm}%
                                {-1explus -.5ex minus -.2ex}%
                                {0.5ex plus .2ex}%
                                {\normalfont\normalsize\bfseries}}
\renewcommand{\subsubsection}{\@startsection{subsubsection}{3}{0mm}%
                                {-1ex plus -.5ex minus -.2ex}%
                                {1ex plus .2ex}%
                                {\normalfont\small\bfseries}}
\makeatother


% -----------------------------------------------------------------------

\begin{document}

\raggedright
\footnotesize

\begin{center}
     \Large{\textbf{Контроллер дверного замка hh24lock}} \\
\end{center}
\begin{multicols}{3}
\setlength{\premulticols}{1pt}
\setlength{\postmulticols}{1pt}
\setlength{\multicolsep}{1pt}
\setlength{\columnsep}{2pt}

\section{Общие положения}

Это руководство является дополнением к инструкции по подключению и программированию кодонаборной панели. 

\section{Совместимость с другими модулями}

Варианты исполнения S2-S4 помимо функции удаленного управления дверным замком также поддерживают подключение внешних датчиков  в составе различных исполнений модуля hh24ups. 

\textbf{ВНИМАНИЕ! контроллеры hh24lock не поддерживают управление внешними реле, а также работу с модулями hh24block и hh24sens.}

\subsection{Совместимость с различными вариантами блоков hh24ups}

\noindent\begin{tabular}{p{2cm}|p{1cm}|p{1cm}|p{1cm}|p{1cm}}
\hline
\texttt{}&\texttt{hh24ups S1}&\texttt{hh24ups S2}&\texttt{hh24ups S3}&\texttt{hh24ups S4}\\
\hline
\textbf{S1}&\multicolumn{1}{c}{\textbf{+}}&\multicolumn{1}{c}{\textbf{-}}&\multicolumn{1}{c}{\textbf{-}}&\multicolumn{1}{c}{\textbf{-}}\\
\textbf{S2}&\multicolumn{1}{c}{\textbf{+}}&\multicolumn{1}{c}{\textbf{+}}&\multicolumn{1}{c}{\textbf{-}}&\multicolumn{1}{c}{\textbf{-}}\\
\textbf{S3}&\multicolumn{1}{c}{\textbf{+}}&\multicolumn{1}{c}{\textbf{-}}&\multicolumn{1}{c}{\textbf{+}}&\multicolumn{1}{c}{\textbf{-}}\\
\textbf{S2}&\multicolumn{1}{c}{\textbf{+}}&\multicolumn{1}{c}{\textbf{+}}&\multicolumn{1}{c}{\textbf{+}}&\multicolumn{1}{c}{\textbf{+}}\\
\hline
\end{tabular}

\textbf{При использовании контроллеров вариантов S2-S4 с модулем hh24ups S1 функционал опроса датчиков недоступен}

\subsection{Совместимость со внешними считывателями Weigand 26|34}

\noindent\begin{tabular}{p{7cm}|p{5cm}}
\hline
\textbf{hh24lock S1}&\multicolumn{1}{c}{\textbf{+}}\\
\textbf{hh24lock S2}&\multicolumn{1}{c}{\textbf{-}}\\
\textbf{hh24lock S3}&\multicolumn{1}{c}{\textbf{-}}\\
\textbf{hh24lock S4}&\multicolumn{1}{c}{\textbf{-}}\\
\hline
\end{tabular}

\section{Функциональные возможности}

\begin{itemize} 
  \item управление замком через сервер hh24lock и внешнее API
  \item удаленный контроль уровня шума в помещении с распознованием типа шума (варианты S2 и S4 при использовании совместно с блоком hh24sens соответствующего исполнения)
  \item удаленный контроль присутствия в помещении (варианты S3 и S4 при использовании совместно с блоком hh24sens соответствующего исполнения)
\end{itemize}

\section{Монтаж контроллера}

Полностью по инструкции кодонаборной панели. В случае использования блока hh24ups с учетом инструкции к блоку. 

\textbf{При монтаже устройств в исполнениях S2-S4 важно не допускать замыкания белого и зеленого проводников на контакты GND или + питания - устройство может выйти из строя.} 


\section{Первое включение устройства}

При первом включении устройство сохраняет для дальнешей работы сведения о Wi-Fi сети, с которой оно будет работать. Для успешного подключения понадобится пароль от Wi-Fi сети и любое другое устройство, имеющее возможность работы с Wi-Fi (например, мобильный телефон или ноутбук).

Для осуществления привязки необходимо:

\begin{itemize}
  \item подключить устройство к контроллеру СКУД согласно схеме выше 
  \item удостовериться в наличии и работоспособности Wi-Fi сети, с которой будет работать устройство
  \item осуществить принудительную перезагрузку устройства 
  \item на втором устройстве в списке сетей Wi-Fi выбрать сеть hh24lock и подключиться к ней
  \item подключение потребует дополнительной авторизации - на мобильном телефоне или ноутбуке появится соответствующее уведомление. В случае отсутствия уведомления можно открыть в браузере сайт https://hh24lock.ru/ 
  \item в появившемся диалоговом окне необходимо выбрать Wi-Fi сеть, с которой будет работать устройство, и ввести пароль к ней
  \item надпись Success укажет на успешное подключение устройства. При этом из списка доступных Wi-Fi сетей исчезнет сеть hh24lock.
\end{itemize}


В случае смены рабочей Wi-Fi сети и/или пароля к ней, устройство потребуется переподключить. 

\section{Принудительная перезагрузка устройства}

Принудительная перезагрузка устройства возможна двумя способами:  

\begin{itemize} 
  \item путем кратковременного отключения устройства от линии питания 12 Вольт
  \item при снятой задней крышке путем нажатия тонким предметом на кнопку сброса, расположенную за наклейкой
\end{itemize}

Успешность перезагрузки устройства индикатируется зеленым сигналом светодиода. 

\section{Возможные неполадки и способы их устранения}

\noindent\begin{tabular}{p{4cm}|p{4cm}}
\texttt{Устройство не передает данные и/или не получает коды открытия двери с сервера}&\texttt{Переподключить устройство к Wi-Fi сети как описано в разделе "Первое включение устройства"}\\
\hline
\texttt{Устройство не передает данные датчиков}&\texttt{Проверить корректность монтажа зеленого и белого проводов и настройки датчиков}\\
\hline
\end{tabular}

\begin{luacode}
local user = require("user")
	function get_login(id)
		tex.print(user.get_login(string.match(id,'%d+')))
	end
	function get_password()
		tex.print(user.get_password())
	end
	function get_master_code()
		tex.print(user.get_master_code())
	end
\end{luacode}

\section{Стандартные настройки}

\subsection{Доступ к hh24lock.ru и API}

\fbox{ 
	ID: \textbf{\jobname }
	Логин: \textbf{\luadirect{get_login(\luastring{\jobname})} }
	Пароль: \textbf{\luadirect{get_password()} }	
}

\subsection{Мастер-код по умолчанию: \textbf{\luadirect{get_master_code()}}}

\end{multicols}
\end{document}
